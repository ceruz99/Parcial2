\documentclass{article}
\usepackage[utf8]{inputenc}
\usepackage[spanish]{babel}
\usepackage{listings}
\usepackage{graphicx}
\graphicspath{ {images/} }
\usepackage{cite}

\begin{document}

\begin{titlepage}
    \begin{center}
        \vspace*{1cm}
            
        \Huge
        \textbf{MANUAL DE USO PARCIAL 2}
            
            
        \vspace{1.5cm}
            
        \textbf{Santiago Pereira Ramírez\\
            Juan Camilo Mazo Castro}
            
        \vfill
            
        \vspace{0.8cm}
            
        \Large
        Despartamento de Ingeniería Electrónica y Telecomunicaciones\\
        Universidad de Antioquia\\
        Medellín\\
        Septiembre de 2021
            
    \end{center}
\end{titlepage}
\newpage
\begin{titlepage}
    \Huge
    \textbf{Pasos a seguir:\\\\}
    \large
    \textbf{
    1.Debe ingresar una imagen en formato .jpg O .png a la carpeta llamada Images que se encuentra en la carpeta CodigoP2Qt.\\
    2.Una vez tenga la imagen lista, puede ejecutar el programa de Qt.\\
    3.El programa le presentará dos opciones, una para obtener la información para presentar una nueva imagen en la matriz de leds y otra para salir del programa.\\
    4.Si selecciona la opción para procesar una nueva imagen, el programa le indicará que ingrese la ruta que tiene la imagen, luego debe presionar la tecla Enter.\\
    5.Una vez realizado lo anterior, el programa habrá finalizado de procesar la nueva imagen y le mostrará nuevamente el menú anterior, si desea hacer lo mismo con otra imagen puede repetir el proceso.\\
    6.Una vez haya finalizado el proceso de una imagen, puede ingresar a la carpeta CodigoP2Qt donde encontrará un archivo .txt llamado "Datos", debe abrirlo.\\
    7.Cuando haya abierto el archivo "Datos.txt" observará que hay una información dentro de él. Debe seleccionarla toda y presionar Ctrl C.\\ 
    8.Ahora debe ingresar a la plataforma Tinkercad donde encontrará el circuito y la matriz de leds.\\
    9.Debe ingresar a la parte del código en Tinkercad y encontrará un apartado que dice "PEGUE AQUÍ".\\
    10.Al encontrar el apartado "PEGUE AQUÍ", dará click en la siguiente linea después del apartado y presionará Ctrl V, pegando así la información que se copió en el paso 7.\\
    11. Al finalizar todos los pasos anteriores ya puede ejecutar el programa de Tinkercad y observará en la pantalla de leds la bandera que desea enseñar.\\
    }
    
\end{titlepage}


\end{document}
