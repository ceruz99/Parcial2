\documentclass{article}
\usepackage[utf8]{inputenc}
\usepackage[spanish]{babel}
\usepackage{listings}
\usepackage{graphicx}
\graphicspath{ {images/} }
\usepackage{cite}

\begin{document}

\begin{titlepage}
    \begin{center}
        \vspace*{1cm}
            
        \Huge
        \textbf{MANUAL DE USO PARCIAL 2}
            
            
        \vspace{1.5cm}
            
        \textbf{Santiago Pereira Ramírez\\
            Juan Camilo Mazo Castro}
            
        \vfill
            
        \vspace{0.8cm}
            
        \Large
        Despartamento de Ingeniería Electrónica y Telecomunicaciones\\
        Universidad de Antioquia\\
        Medellín\\
        Septiembre de 2021
            
    \end{center}
\end{titlepage}

\newpage

\section{Pasos a seguir:}\label{contenido}
  
  \begin{enumerate}
        \item Ingrese al entorno de desarrollo Qtcreator y escoja o busque el proyecto llamado CodigoP2Qt.\\
        
         \item Ejecute el programa de Qt.\\ 
         
         \item El programa le presentará dos opciones, una para obtener la información para presentar una nueva imagen en la matriz de leds y otra para salir del programa (debe de ingresar el numero de la opcion).\\
         
         \item Si selecciona la opción para procesar una nueva imagen, el programa le indicará que ingrese la ruta donde esta la imagen, luego debe presionar la tecla Enter.\\
         
         \begin{enumerate}
         
         \item Una vez realizado lo anterior, el programa habrá finalizado el procesamiento de la nueva imagen y le mostrará nuevamente el menú anterior, si desea hacer lo mismo con otra imagen puede repetir el proceso desde el paso 3, de otra forma escoja la opcion para salir del programa.\\
    
        \item Una vez haya finalizado, puede ingresar a la carpeta CodigoP2Qt donde encontrará un archivo .txt llamado "Datos", debe abrirlo.\\
        
        \item Cuando haya abierto el archivo "Datos.txt",  observará que hay una información dentro de él. Debe seleccionarla toda y presionar Ctrl C.\\ 
        
        \item Entre al siguiente link a travez de su navegador para copiar el proyecto https://www.tinkercad.com/things/1ipoJc5pro4 , despues de estar en la pestaña del proyecto de Tinkercad presione donde dice copiar y modificar(creara una copia del proyecto para que lo use a gusto).\\

       \item  Cuando ya pueda hacer modificaciones, debe ingresar a la parte del código en Tinkercad y encontrará un apartado que dice "PEGUE AQUÍ".\\

       \item Al encontrar el apartado "PEGUE AQUÍ", dará click en la siguiente linea después del apartado y presionará Ctrl V, pegando así la información que se copió en el paso a.\\
       
       \item Al finalizar todos los pasos anteriores presione donde dice "Iniciar simulación" y observará en la matriz de leds la bandera que desea enseñar.\\
       
         \end{enumerate}

    \end{enumerate}  

\end{document}
