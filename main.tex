\documentclass{article}
\usepackage[utf8]{inputenc}
\usepackage[spanish]{babel}
\usepackage{listings}
\usepackage{graphicx}
\graphicspath{ {images/} }
\usepackage{cite}

\begin{document}

\begin{titlepage}
    \begin{center}
        \vspace*{1cm}
            
        \Huge
        \textbf{Documento para el parcial 2}
            
        \vspace{0.5cm}
        \LARGE
        Subtítulo
            
        \vspace{1.5cm}
            
        \textbf{Nombres y Apellidos del autor}
            
        \vfill
            
        \vspace{0.8cm}
            
        \Large
        Despartamento de Ingeniería Electrónica y Telecomunicaciones\\
        Universidad de Antioquia\\
        Medellín\\
        Marzo de 2021
            
    \end{center}
\end{titlepage}

\tableofcontents
\newpage
\section{Comprension del problema}\label{intro}
Se debera a aartir de una imagen de una bandera,representarla a travez de una matriz de leds RGBs en el programa tinkerkad y el entorno de desarrollo Qt creator

\section{Elementos a emplear} \label{contenido}
Para la practica se debera de utilizar clases y objetos para el empleo de los algortimos que se deben de utilizar y las demas funciones y metodos para el correcto empleo del programa, asi mismo deberemos de trabajar con arduino en la plataforma tinkerkad.


\section{Programacion} \label{contenido}
--creacion del arhivo de Qt y hacer pruebas con los conociemintos  acerca de la manipulacion de imagenes 
despues de esto se podra crear la clase que nos permitira crear los metodos de muestreo para redimensionar la imagen y llevar los datos hacia un documento.

--Empezar a crear el argoritmo de submuestreo y sobremuestreo,hacer las diversas pruebas para verificar que se este dessarrollando de buena manera.

-crear el tinkercad y hacer las conexion.

-- crear un metodo que reciba los datos del metodo de muestreo.

--Utilizar la funcion para prender los leds RGBs en arduino y copiar la informacion en el.

--






\end{document}
